%% start of file `template.tex'.
%% Copyright 2006-2015 Xavier Danaux (xdanaux@gmail.com).
%
% This work may be distributed and/or modified under the
% conditions of the LaTeX Project Public License version 1.3c,
% available at http://www.latex-project.org/lppl/.

\documentclass[10pt,a4paper,sans]{moderncv}
% moderncv themes
\moderncvstyle{casual}
\moderncvcolor{blue}                               
\usepackage[utf8]{inputenc}                       
\usepackage[scale=0.85]{geometry}
\renewcommand{\refname}{Publicaciones}

% personal data
\name{Horacio}{Chacón Torrico}
\title{Médico | Maestro en Informática Biomédica en Salud Global}       
\address{Av. República de Panamá 257 Barranco}{Lima,Peru}
\phone[mobile]{+51~996~061~993}    
\phone[fixed]{+51~(1)~693~5905} 
\email{horaciochacon89@gmail.com}
\homepage{horaciochacon.rbind.io}  
\social[linkedin]{horacio-chacón-torrico} 
\social[twitter]{horacio89}
\social[github]{horaciochacon}                 
\photo[64pt][0.4pt]{imagen}                  
%            

% to show numerical labels in the bibliography (default is to show no labels)
\makeatletter\renewcommand*{\bibliographyitemlabel}{\@biblabel{\arabic{enumiv}}}\makeatother
%   to redefine the bibliography heading string ("Publications")
%\renewcommand{\refname}{Articles}

%-------------------------------------------------------------
%            content
%-------------------------------------------------------------
\begin{document}

\makecvtitle

\section{Educación}

\cventry{2017--2019}{Msc}
{Universidad Peruana Cayetano Heredia}{Lima}{Peru}{Maestro en Informática Biomédica en Salud Global con Mención en Informática en Salud.}
\cventry{2008--2015}{MD}
{Universidad Científica del Sur}{Lima}{Peru}{Médico Cirujano.}  

\section{Experiencia}
    \subsection{Experiencia en Investigación}
        
        \cventry{2020}{Investigador Asociado}{Universidad Científica del Sur}{Lima, Peru}{}{Plaza de investigador asociado.}
        
        \cventry{2020}{\href{https://ctivitae.concytec.gob.pe/appDirectorioCTI/UploadFotoPath.do?tipo=visualizar_archivo_codigo_renacyt&id_investigador=27771&ruta=/documents/renacyt/constancias/P0027771.pdf}{Calificación RENACYT: Maria Rostworoski II}}{CONCYTEC}{Lima, Peru}{}{}
        
        \cventry{2016--2020}{Investigador}{Universidad Peruana Cayetano Heredia}{Lima, Peru}{}{
            \begin{itemize}
            \item \href{https://www.youtube.com/watch?v=MAzaYVm1VJY&feature=youtu.be}{Adaptación de un aplicativo móvil para el monitoreo y tamizaje poblacional de COVID-19 en Loreto.}
            \item Mamás del Río: Mejorando la salud materna e infantil en áreas rurales en la Amazonía Peruana.
            \end{itemize}
            }


    \subsection{Experiencia en Docencia}
        \cventry{2019--2020}{Docente}
        {Universidad Cientifica del Sur}{Lima, Peru}{}{}
        
    \subsection{Experiencia Laboral}
        \cventry{2019--2020}{Médico SERUMS}
        {Posta Médica Escuela de Policía Puente Piedra}{Lima, Peru}{}{}
        
        \cventry{2016--2019}{Coordinador y Director Tecnológico}
        {Universidad Peruana Cayetano Heredia}{Peru}{}
        {Proyecto Mamás del Río.}
        
        \cventry{2016--2017}{Consultor}
        {Ministerio de Salud - Iniciativa Bloomberg}{Lima, Peru}{}
        {Evaluación y entrenamiento de nuevo Sistema Nacional de Defunciones (SINADEF).}
        
        \cventry{2015}{Consultor}
        {Instituto Nacional de Salud}{Lima, Peru}{}
        {Consultoría para la gestión de implementación del sistema informático de laboratorio NetLab versión 2.0.} 
        
        \cventry{2015--2016}{Analista Médico}
        {Sociedad Operadora de Salud}{Lima, Peru}{}{Análisis y procesamiento de información de historias clínicas electrónicas de un complejo hospitalario informatizado.} 

    \subsection{Proyectos}
        \cventry{2019}{ENDES.PE}{\href{https://github.com/horaciochacon/ENDES.PE}{Desarrollo de paquete para el análisis y manipulación de la ENDES Perú.}}{}{}{}

\section{Contribución a Políticas Públicas}
    \cventry{2017}{Soporte Técnico}{Ministerio de Salud}{Loreto, Peru}{}{\href{ftp://ftp2.minsa.gob.pe/normaslegales/1980/RM_N\%C2\%B0594-2017-MINSA.pdf}{Modelo de Atención de Salud Integral e Intercultural de las Cuencas de los Ríos Tigre, Marañón y Chambira en la Región Loreto 2017 – 2021.}}

\section{Premios y Distinciones}
    \cventry{2020}{Beca}{Universidad Científica del Sur}{Lima, Perú}{}{Beca Cabieses: concurso de investigador asociado 2020.}
    
    \cventry{2018}{Beca curso}{Centro Latinoamericano de Formación Interdisciplinaria}{Buenos Aires, Argentina}{}{Salud y ambiente: integrando fenómenos espacio-temporales bajo un enfoque frecuentista y bayesiano.}

\nocite{*}
\bibliographystyle{unsrt}
\bibliography{publications}                     

\section{Habilidades Tecnológicas}
    \cvdoubleitem{\textbf{Advanzado}}{SPSS, R, Markdown}{\textbf{Intermedio}}{GIS, STATA, SQL, \LaTeX, Git}
    \cvdoubleitem{\textbf{Básico}}{Bash, Linux, Html, CSS}{\textbf{Otros}}{Gestores bibliográficos}

\section{Idiomas}
    \cvitemwithcomment{Español}{Fluido}{Lengua nativa}
    \cvitemwithcomment{Inglés}{Avanzado, TOEFL iBT (103)}{Escritura, lectura, oral}
    \cvitemwithcomment{Alemán}{Intermedio, Spachdiplom I \& II, B2.2}{Lectura, oral}


\section{Referencias}
\begin{cvcolumns}
  \cvcolumn{}{\textbf{Magaly Blas, MD, MPH, PhD}\\ University of Washington \\ Cayetano Heredia  University \\ \emailsymbol \href{mailto:magaly.blas@upch.pe}{magaly.blas@upch.pe}}
  
  \cvcolumn{}{\textbf{Joseph Zunt, MD,MPH}\\ Department of Neurology \\ University of Washington \\\emailsymbol \href{mailto:jzunt@uw.edu}{jzunt@uw.edu}}
  
\end{cvcolumns}



\end{document}

