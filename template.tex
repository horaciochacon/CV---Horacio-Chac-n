%% start of file `template.tex'.
%% Copyright 2006-2015 Xavier Danaux (xdanaux@gmail.com).
%
% This work may be distributed and/or modified under the
% conditions of the LaTeX Project Public License version 1.3c,
% available at http://www.latex-project.org/lppl/.

\documentclass[11pt,a4paper,sans]{moderncv}
% moderncv themes
\moderncvstyle{casual}
\moderncvcolor{blue}                               
\usepackage[utf8]{inputenc}                       
\usepackage[scale=0.75]{geometry}

% personal data
\name{Horacio}{Chacón Torrico}
\title{MD | MSc }                              
\address{Av. República de Panamá 257 Barranco}{Lima,Peru}
\phone[mobile]{+51~996~061~993}    
\phone[fixed]{+51~(1)~693~5905} 
\email{hchacont@uw.edu | horaciochacon89@gmail.com}
\homepage{horaciochacon.rbind.io}  
\social[linkedin]{horacio-chacón-torrico} 
\social[twitter]{horacio89}
\social[github]{horaciochacon}                 
\photo[64pt][0.4pt]{Imagen}                  
%            

% to show numerical labels in the bibliography (default is to show no labels)
\makeatletter\renewcommand*{\bibliographyitemlabel}{\@biblabel{\arabic{enumiv}}}\makeatother
%   to redefine the bibliography heading string ("Publications")
%\renewcommand{\refname}{Articles}

%-------------------------------------------------------------
%            content
%-------------------------------------------------------------
\begin{document}

\makecvtitle

\section{Education}

\cventry{2017--2019}{Msc}
{Cayetano Heredia University}{Lima}{Peru}{MSc. in Biomedical Informatics in Global Health major in Healthcare Informatics, Thesis title: “Usability and feasibility evaluation of an mHealth application for community health workers in remote Amazon Peruvian communities”}
\cventry{2008--2015}{MD}
{Scientific University of the South}{Lima}{Peru}{General Medicine}  

\section{Experience}
    \subsection{Research Experience}
        
        \cventry{2020}{Researcher}{Cayetano Heredia University}{Lima, Peru}{}{Adaptation of an mHealth application for the screening and monitoring of COVID-19 population in Loreto. Grant accepted (CONCYTEC): 04/26/2020. PI: Dr. Magaly Blas. Duration: 3 months}
        
        \cventry{2016-2019}{Researcher}{Cayetano Heredia University}{Lima, Peru}{}{Mamas del Rio multicomponent maternal and neonatal community health worker based intervention in Amazon remote native communities. PI: Dr. Magaly Blas}
        
        \cventry{2012}{Research Exchange}{TU-Dresden  Smell \& Taste Clinic}{Dresden, Germany}{}{Advisor: Dr. Thomas Hummel}


    \subsection{Teaching Experience}
        \cventry{2019 - 2020}{Teaching Assitant}
        {Universidad Cientifica del Sur}{Lima, Peru}{}
        {Evidence Based Medicine (EBM). Core research fundaments in medicine and health.}
        
        \cventry{2020}{Lecturer}
        {Universidad Cientifica del Sur}{Lima, Peru}{}
        {Thesis Seminars. Core research fundaments in medicine and health applyed to a final thesis project}
        
    \subsection{Work Experience}
        \cventry{2019--2020}{General Practitioner}
        {Puente Piedra Police Health Post}{Lima, Peru}{}
        {Primary care health work attending patients as part of the rural-urban marginal health service. Aside of recuperative medicine, preventive and promotional health activities were done.}
        
        \cventry{2016--2019}{Researcher \& Tecnology director}
        {Universidad Peruana Cayetano Heredia}{Peru}{}
        {Mamás del Río: Multicomponent Community health worker driven intervention for the reduction of neonatal mortality was designed and piloted for a rural setting. Main responsibilities included, funding acquisition, design of mHealth platform, implementation, project deadline followup, field-trips to execution site and stakeholder involvement meetings. Funding: Grand Challenges Canada and Peruvian National Science and Technology Councyl.}
        
        \cventry{2016-2017}{Consultant}
        {Ministry of Health}{Lima, Peru}{}
        {Evaluation and training of the new electronic national death certificate reporting system.} 
        
        \cventry{2015}{Consultant}
        {National Institute of Health}{Lima, Peru}{}
        {Implementation and information management consulting for the design of the Public Health Laboratory Information System (NetLab V2.0).} 
        
        \cventry{2015--2016}{Medical data analyst}
        {Sociedad Operadora de Salud}{Lima, Peru}{}{Data analysis and processing of the first fully implemented Electronic Health Record system in Peru.} 

    \subsection{Side Projects}
        \cventry{2019}{ENDES.PE}{}{}{}
        {Development of an R package for the handling of peruvian DHS survey data (https://github.com/horaciochacon/ENDES.PE)}

\section{Public Policy Contribution}
    \cventry{2017}{Technical Support}{Ministry of Health}{Loreto, Peru}{}{Integral and intercultural health care model for the Pastaza, Corrientes, Tigre, Marañon, and Chambira river basins in the Loreto region (2017-2021).}

\section{Awards}
    \cventry{2018}{Scholarship}{Interdiciplinary Latinamerican training center}{Buenos Aires, Argentina}{}{A full cost scholarship was awarded for the 2 week course: "Health and enviroment: integrating spatio-temporal phenomena by means of a frecuentist and bayesian approach".}
    \cventry{2016}{Awarded}{Global Innovation through science and technology demo day - 2nd place}{VentureWell and US Department of State}{}{Second place by pitching the Mamás del Río mHealth innovation and business model.}


\nocite{*}
\bibliographystyle{unsrt}
\bibliography{publications}                     

\section{Computer skills}
    \cvdoubleitem{\textbf{Advanced}}{SPSS, R, Markdown}{\textbf{Intermediate}}{GIS, STATA, SQL, \LaTeX, Git}
    \cvdoubleitem{\textbf{Basic}}{Bash, Linux, Html, CSS}{\textbf{Others}}{Reference Managers}

\section{Languages}
    \cvitemwithcomment{Spanish}{Fluent}{Native speker}
    \cvitemwithcomment{English}{Advanced, TOEFL iBT (103)}{Read, writen, spoken}
    \cvitemwithcomment{German}{Intermediate, Spachdiplom I \& II, B2.2}{Read, spoken}


\section{References}
\begin{cvcolumns}
  \cvcolumn{}{\textbf{Magaly Blas, MD, MPH, PhD}\\ University of Washington \\ Cayetano Heredia  University \\ \emailsymbol \href{mailto:magaly.blas@upch.pe}{magaly.blas@upch.pe}}
  
  \cvcolumn{}{\textbf{Joseph Zunt, MD,MPH}\\ Department of Neurology \\ University of Washington \\\emailsymbol \href{mailto:jzunt@uw.edu}{jzunt@uw.edu}}
  
\end{cvcolumns}



\end{document}

